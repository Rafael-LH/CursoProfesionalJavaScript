Event Loop
El Event Loop hace que Javascript parezca ser multihilo a pesar de que corre en un solo proceso.

Javascript se organiza usando las siguientes estructuras de datos:

Stack / pila. Va apilando de forma organizada las diferentes instrucciones que se llaman.
              Lleva así un rastro de dónde está el programa, en que punto de ejecución nos encontramos.
Memory Heap. De forma desorganizada se guarda información de las variables y del scope.
Schedule Tasks. Aquí se agregan a la cola, las tareas programadas para su ejecución.
Task Queue / cola de tareas. Aquí se agregan las tares que ya están listas para pasar al stack y ser ejecutadas.
                             El stack debe estar vacío para que esto suceda.
MicroTask Queue (cola secundaria donde se agregan las tareas las cuales contienen promesas). Aquí se agregan las promesas.
                Esta Queue es la que tiene mayor prioridad.
El Event Loop es un loop que está ejecutando todo el tiempo y pasa periódicamente revisando las queues y el stack moviendo tareas entre estas dos estructuras.