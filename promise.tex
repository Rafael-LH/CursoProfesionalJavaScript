Promesas
Para crear las promesas usamos la clase Promise. El constructor de Promise recibe un sólo argumento,
un callback con dos parámetros, resolve y reject.
resolve es la función a ejecutar cuando se resuelve y reject cuando se rechaza.

El async/await es sólo syntax sugar de una promesa, por debajo es exactamente lo mismo.

La clase Promise tiene algunos métodos estáticos bastante útiles:

Promise.all. Da error si una de las promesas es rechazada. (con all corremos todas las promesas que tenemos le pasamos como 
parametro en un arrgle todas las promesas es decir Promise.all([promesa1, promesa2]))

Promise.race. Regresa sólo la promesa que se resuelva primero. 
igual que con promise all le mandamos todas nuestras promesas como parametro Promise.race([promesa1, promesa2]))